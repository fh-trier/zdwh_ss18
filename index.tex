% Dokumentanklasse: a4paper, 14pt
% Beschreibung:     Dokumentenformat
% Option:           extraarticle - ?
\documentclass[a4paper, 14pt]{extarticle}

% Paket:            geometry
% Beschreibung:     A4 Seitenabstände
% Option:           
\usepackage{geometry}
\geometry{
  a4paper,            % Papierformat
  top=2cm,            % Abstand Kopfseite
  bottom=2cm,         % Abstand Fußseite
  left=2cm,           % Abstand Linkeseite
  right=2cm,          % Abstand Rechteseite
%  nohead=TRUE,        % Keine Kopfzeilen
%  nofoot=TRUE,        % Keine Fußzeilen
%  marginparwidth=0cm, % Seitenabstand rechts
%  marginparsep=0cm,   % 
%  width=17cm
}

% Paket:            ansmath
% Beschreibung:     Zum darstellen von mathematischen Formeln
\usepackage{amsmath}

% Paket:            ngerman
% Beschreibung:     Deutsche Rechtschreibung
% Option:           babel - Sibentrennung
\usepackage[ngerman]{babel}

% Paket:            utf8
% Beschreibung:     Stellt Umlaute richtig dar
% Option:           inputenc - Erlaubt die Darstellung der gleichen Zeichen (Character) wie sie in stdin überliefert werden
\usepackage[utf8]{inputenc}

% Paket:            makeindex
% Beschreibung:     Ermöglicht das Indexieren von Wörter und den Befehl \printindex um den Index auszugeben
\usepackage{makeidx}
\makeindex

% Paket:            natbib
% Beschreibung:     Für Zitate
% Option:           round - ?
%\usepackage[round]{natbib}

% Paket:            fancyhdr
% Beschreibung:     Ermöglich ein generelles Seitenlayout ein zu stellen mit Kopf und Fußzeile.
\usepackage{fancyhdr}

% Paket:            graphicx
% Beschreibung:     Einbinden von Bildern
% Option:
\usepackage{graphicx}

% Paket:            enumitem
% Beschreibung:     Zeilenabstände bei Aufzählungen definieren
% Option:
\usepackage{enumitem}

% Paket:            pdflscape
% Beschreibung:     Ermöglicht Seiten horizontal darzustellen
% Option:           \begin{landscape} \end{landscape}
\usepackage{pdflscape}

% Paket:            float
% Beschreibung:     Zum Ausrichten von Tabellen und Spalten bzw. deren Zentrierung
% Option:
% Restriktion:      Muss von Paket hyperref geladen werden. Ansonsten funktioniert das Paket nicht.
\usepackage{float}

% Paket:            multirow
% Beschreibung:     Zum kombinieren mehrerer Zellen einer Tabelle
% Option:
\usepackage{multirow}

% Paket:            courier
% Beschreibung:     Lädt das Paket courier für Schriftarten mit fester Breite.
% Befehle:          \ttfamily     Aktiviert Courier füt Tabellen bzw. generelle begin-Blöcke
\usepackage{courier}

% Paket:            appendix
% Beschreibung:     Das Paket dient dazu, ausschließlich das Thema einer Überschrift in das Inhaltsverzeichnis zu überführen
% Option:           appendix - Überführt die Überschriften des Anhangs richtig ins das Inhaltsverzeichnis
\usepackage[titletoc]{appendix}

% Paket:            setspace
% Beschreibung:     Setz über die optionen den Zeilenabstand
% Optionen:         Möglicher Zeilenabstand
%                   singlespacing = 1,0
%                   onehalfspacing = 1,5
%                   doublespacing = 2,0
% Restriktion:      Muss von Paket hyperref geladen werden. Ansonsten funktioniert das Paket nicht.
\usepackage[onehalfspacing]{setspace}

% Paket:            courier
% Beschreibung:     Läd die Schriftart courier. Courier besitzt feste Schriftbreiten.
% Option:
\usepackage{courier}

% Packet:           Hyperref
% Beschreibung:     Importiert hyperref um Querverweise mittels \hyperref zu erzeugen.
\usepackage{hyperref}
\hypersetup{
  pdftitle={Tutorium},
  pdfauthor={Markus Pesch},
  pdfsubject={Datawarehouse}
}

% Packet:           Minted
% Beschreibung:     Dient zum highlining von Quellcode wie beispielsweise Java, Bash oder Python.
% Option/en:             
%   autogobble:       Leerzeichen zwischen linken Rand und Sourcecode einrücken bzw. weg schneiden. 
%   breaklines:       Automatische Zeilenumbrüche
%   cache:            de- oder aktiviert den cache um Sourcecode zwischen zu speichern und so das PDF schneller zu erzeugen
%   cachedir:         Definiert den Pfad zum cache, an dem minted seine Daten zwischen speichern kann
%   fontfamily:       Die Schriftart die benutzt werden soll. tt, courier und helvetica sind vordefiniert.
%   fontsize:         Die Schriftgröße die benutzt werden soll. Beispielsweise fontsize=\footnotesize
%   linenos:          Zeilennummern
%   keywordcase:      Änderung der Buchstaben. Takes lower, upper, or capitalize.
%   showspaces:       Blendet Leerzeichen ein
\usepackage[cache=true, cachedir=/tmp]{minted}

% usemintedstyle
% Gebe 'pygmentize -L styles' im Terminal ein um alle verfügbaren styles anzuzeigen.
\usemintedstyle{tango}

% newminted
% Definiere neue aliase um einmalig ein highlighting pro Sprache zu deklarieren
% \newminted{<makroname>}{optionen} ist verfügbar unter <makroname>_code
\newminted{awk}{autogobble=true, breaklines=true, linenos=true}
\newminted{json}{autogobble=true, breaklines=true, linenos=true}
\newminted{r}{autogobble=true, breaklines=true, linenos=true}
\newminted{sql}{autogobble=true, breaklines=true, linenos=true}


% newmintedfile
% Definiere neue Makros um automatisch Sourcecode aus Dateien zu highlighten.
\newmintedfile[inputawk]{awk}{autogobble=true, breaklines=true, linenos=true, keywordcase=upper}
\newmintedfile[inputjson]{json}{autogobble=true, breaklines=true, linenos=true}
\newmintedfile[inputsql]{sql}{autogobble=true, breaklines=true, linenos=true, keywordcase=upper}
\newmintedfile[inputr]{r}{autogobble=true, breaklines=true, linenos=true}

% newmintinline
% Definiere neues Makro um Sourcecoude einzeiler zu highlighten
\newmintinline{awk}{awkcode}
\newmintinline{json}{jsoncode}
\newmintinline{r}{rcode}
\newmintinline{sql}{sqlcode}


% Packet:           tabularx
% Beschreibung:     Werden Tabellen mit diesem Paket erstellt, ist es möglich Zeilenumbrüche in einer Zelle zu erzeugen
\usepackage{tabularx}

% Packet:           framemethod
\usepackage[framemethod=tikz]{mdframed}
\mdtheorem[
  linecolor=red,
  frametitlefont=\sffamily\bfseries\color{white},
  frametitlebackgroundcolor=red,
]{warn-popup}{Warnung}[subsection]

\mdtheorem[
  linecolor=orange,
  frametitlefont=\sffamily\bfseries\color{white},
  frametitlebackgroundcolor=orange,
]{info-popup}{Information}[subsection]

% Start des Dokuments
\begin{document}

  % Fetch Commit ID and Date
  \immediate\write18{./git-info.sh commit > git-id.tmp}
  \immediate\write18{./git-info.sh date > git-date.tmp}
  \immediate\write18{./git-info.sh url > git-url.tmp}

  % Importiere weitere .tex Dokumente
  \begin{titlepage}
  \begin{center}
    \begin{large}
      Klausur SS/WSXX, Tutorium SS/WSXX oder Zusammenfassung 
    \end{large}
    
    \begin{huge}
      \begin{singlespace}
            \textbf{Modul}
      \end{singlespace}
    \end{huge}

    \vspace{0.5cm}

    \begin{figure}[h]
      \centering
      \includegraphics[width=0.85\textwidth]{img//logo.png}
      \label{img:fh-trier-logo}
    \end{figure}

    \vspace{2cm}
    \begin{large}
      \textit{Markus Pesch} \\
      \textit{peschm@hochschule-trier.de}
    \end{large}
    \vspace{2cm}
    
    Latex Quellcode auf \input{./git-url.tmp} \\
    basierend auf git commit \input{git-id.tmp} vom \input{git-date.tmp}
    
    
  \end{center}
\end{titlepage}

  \pagebreak

  % Pagestyle
  % Setze das Seitenlayout auf leer um Fuß und Kopfzeilen zu unterdrücken
  \pagestyle{empty}

  % Agenda
  \tableofcontents

  % Pagestyle
  % Setze das Seitenlayout auf fancyhdr um Fuß- und Kopfzeilen zu setzen
  \pagestyle{fancy}

  % Löscht alle Kopf- und Fußzeilen des pagestyles fancyhdr
  \fancyhf{}

  % Fuß- und Kopfzeile des Paketes fancyhdr
  % [L] - Linkeseite      [O] - Ungerade Seitenzahlen         [LE,LO] - Linkeseite, Gerade- und Ungerade Seitenanzahlen
  % [C] - Mitte           [E] - Gerade Seitenanzahlen         [CE]    - Seitenmitte, nur gerade Seitenanzahlen
  % [R] - Rechteseite                                         [RO]    - Rechteseite, nur ungerade Seitenanzahlen
  % \fancyhead    Kopfzeile
  % \fancyfoot    Fußzeile
  \fancyhead[L]{\rightmark}
  \fancyhead[R]{\includegraphics[width=4cm]{img/logo.png}}
  \fancyfoot[L]{Tutorium XX SSXX}
  \fancyfoot[C]{}
  \fancyfoot[R]{Seite \thepage}

  % Pixelstärke der Kopf- und Fußzeilenlinie
  \renewcommand{\headrulewidth}{1pt}
  \renewcommand{\footrulewidth}{1pt}

  % Setze die Seitenbeginn zurück
  \setcounter{page}{1}

  % Importiere weitere .tex Dokumente
  \section{Übung}
\label{sec:uebung_01}

% ##########################################################################
% ############################### Aufgabe 01 ###############################
% ##########################################################################
\subsection{Aufgabe}
\label{sec:uebung_01.aufgabe_01}
Starten Sie das Skript.

\subsubsection*{Lösung}
\label{sec:uebung_01.aufgabe_01.loesung}

\subsubsection*{Boxen}
\begin{warn-popup}
  Tabelle fehlt!
\end{warn-popup}

\begin{info-popup}
  Tabelle fehlt!
\end{info-popup}

\subsubsection*{Minted}
SQL
\begin{sqlcode}
  SELECT *
  FROM tab;
\end{sqlcode}

AWK
\begin{awkcode}
  if NR > 3 {
    print $0
  }
\end{awkcode}


\inputsql{sql/test.sql}

\end{document}
