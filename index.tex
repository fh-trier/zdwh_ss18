% Dokumentanklasse: a4paper, 14pt
% Beschreibung:     Dokumentenformat
% Option:           extraarticle - ?
\documentclass[a4paper,14 pt]{extarticle}

% Paket:            a4paper
% Beschreibung:     A4 Seitenabstände
% Option:           geometry
\usepackage[a4paper]{geometry}

% Paket:            ansmath
% Beschreibung:     Zum darstellen von mathematischen Formeln
\usepackage{amsmath}

% Paket:            ngerman
% Beschreibung:     Deutsche Rechtschreibung
% Option:           babel - Sibentrennung
%\usepackage{ngerman}
\usepackage[ngerman]{babel}

% Paket:            utf8
% Beschreibung:     Stellt Umlaute richtig dar
% Option:           inputenc - Erlaubt die Darstellung der gleichen Zeichen (Character) wie sie in strin überliefert werden
\usepackage[utf8]{inputenc}

% Paket:            makeindex
% Beschreibung:     Ermöglicht das Indexieren von Wörter und den Befehl \printindex um den Index auszugeben
\usepackage{makeidx}
\makeindex

% Paket:            natbib
% Beschreibung:     Für Zitate
% Option:           round - ?
%\usepackage[round]{natbib}

% Paket:            fancyhdr 
% Beschreibung:     Ermöglich ein generelles Seitenlayout ein zu stellen mit Kopf und Fußzeile.
\usepackage{fancyhdr}

% Paket:            graphicx
% Beschreibung:     Einbinden von Bildern
% Option:           
\usepackage{graphicx}

% Paket:            float
% Beschreibung:     Zum Ausrichten von Tabellen und Spalten bzw deren zentrierung
% Option:
% Restriktion:      Muss von Paket hyperref geladen werden. Ansonsten funktioniert das Paket nicht.
\usepackage{float}

% Paket:            appendix
% Beschreibung:     Das Paket dient dazu, ausschließlich das Thema einer Überschrift in das Inhaltsverzeichnis zu überführen
% Option:           appendix - Überführt die Überschriften des Anhangs richtig ins das Inhaltsverzeichnis
\usepackage[titletoc]{appendix}

% Paket:			setspace
% Beschreibung:		Setz über die optionen den Zeilenabstand
% Optionen:			Möglicher Zeilenabstand
%					singlespacing = 1,0
%					onehalfspacing = 1,5
%					doublespacing = 2,0
% Restriktion:      Muss von Paket hyperref geladen werden. Ansonsten funktioniert das Paket nicht.   
\usepackage[onehalfspacing]{setspace}

% Packet:           Hyperref
% Beschreibung:     Importiert hyperref um Querverweise mittels \hyperref zu erzeugen.
\usepackage[]{hyperref}
\hypersetup{
	pdftitle={Reliable and scalable messaging in Microservices},
	pdfauthor={Markus Pesch},
	pdfsubject={Bachelorthesis}
}

% Packet:           Minted
% Beschreibung:     Dient zum highlining von Quellcode wie beispielsweise Java, Bash oder Python. 
\usepackage{minted}
\usemintedstyle{emacs}

% Packet:           tabularx
% Beschreibung:     Werden Tabellen mit diesem Paket erstellt, ist es möglich Zeilenumbrüche in einer Zelle zu erzeugen
\usepackage{tabularx}

% Paket:            biblatex
\usepackage[
    style=authoryear-icomp,    % Zitierstil
    isbn=false,                % ISBN nicht anzeigen, gleiches geht mit nahezu allen anderen Feldern
    pagetracker=true,          % ebd. bei wiederholten Angaben (false=ausgeschaltet, page=Seite, spread=Doppelseite, true=automatisch)
    maxbibnames=50,            % maximale Namen, die im Literaturverzeichnis angezeigt werden (ich wollte alle)
    maxcitenames=3,            % maximale Namen, die im Text angezeigt werden, ab 4 wird u.a. nach den ersten Autor angezeigt
    autocite=inline,           % regelt Aussehen für \autocite (inline=\parancite)
    block=space,               % kleiner horizontaler Platz zwischen den Feldern
    backref=true,              % Seiten anzeigen, auf denen die Referenz vorkommt
    backrefstyle=three+,       % fasst Seiten zusammen, z.B. S. 2f, 6ff, 7-10
    date=short,                % Datumsformat
    backend=biber
]{biblatex}
\setlength{\bibitemsep}{1em}     % Abstand zwischen den Literaturangaben
\setlength{\bibhang}{2em}        % Einzug nach jeweils erster Zeile

\addbibresource{bibliothek.bib}


% Packet:           glossaries
% Beschreibung:     Glossar einbinden und Glossarbefehle bereitstellen 
% Option:           Gebe Glossar auch als section im Inhaltsverzeichnis aus     
\usepackage[toc,section=section]{glossaries}
\makeglossaries
% Glossareinträge
\newglossaryentry{gls:2-trier-architektur}
{
    name={2-Trier-Architektur},
    description={Eine 2-Tier-Architektur beschreibt, dass ein Client direkt mit dem Server kommuniziert. Der Server stellt eine Applikationsschicht und eine Datenbankschicht bereit. Ein Beispiel für eine 2-Tier-Architektur wären PC-Spiele bei denen die Spieler im Multiplayer spielen. Die Kommunikation jedes Spielers als Client erfolgt über einen Server mit dem sich alle Teilnehmer verbinden},
    plural={2-Trier-Architekturen}
}


% Acronyme
\newacronym{api}{API}{Application Programming Interface}
\newacronym{cli}{CLI}{Commando Line Interface}


% Start des Dokuments
\begin{document}
     
    % Importiere weitere .tex Dokumente
	\begin{titlepage}
  \begin{center}
    \begin{large}
      Klausur SS/WSXX, Tutorium SS/WSXX oder Zusammenfassung 
    \end{large}
    
    \begin{huge}
      \begin{singlespace}
            \textbf{Modul}
      \end{singlespace}
    \end{huge}

    \vspace{0.5cm}

    \begin{figure}[h]
      \centering
      \includegraphics[width=0.85\textwidth]{img//logo.png}
      \label{img:fh-trier-logo}
    \end{figure}

    \vspace{2cm}
    \begin{large}
      \textit{Markus Pesch} \\
      \textit{peschm@hochschule-trier.de}
    \end{large}
    \vspace{2cm}
    
    Latex Quellcode auf \input{./git-url.tmp} \\
    basierend auf git commit \input{git-id.tmp} vom \input{git-date.tmp}
    
    
  \end{center}
\end{titlepage}

    \pagebreak
    
    % Pagestyle
    % Setze das Seitenlayout auf leer um Fuß und Kopfzeilen zu unterdrücken
    \pagestyle{empty}
    
    % Importiere weitere .tex Dokumente
    \hspace{0pt}
\vfill
\LARGE{Verfasser}

\begin{table}[H]
	\label{table:verfasser}
	\normalsize
	\begin{tabular}{lll}
		\textbf{Name} 	& \textbf{E-Mail}     		& \textbf{Studiengang}	\\
		Markus Pesch 	& \url{peschm@fh-trier.de}  & Wirtschaftsinformatik	\\
	\end{tabular}
	%\caption{Transaktionsbeispiel - At leat once}
\end{table}

\hspace{0pt}
\vfill
    \include{agenda//agenda}
    
    % Pagestyle
    % Setze das Seitenlayout auf fancyhdr um Fuß- und Kopfzeilen zu setzen
    \pagestyle{fancy}
    
    % Löscht alle Kopf- und Fußzeilen des pagestyles fancyhdr
    \fancyhf{}
    
    % Fuß- und Kopfzeile des Paketes fancyhdr
    % [L] - Linkeseite      [O] - Ungerade Seitenzahlen         [LE,LO] - Linkeseite, Gerade- und Ungerade Seitenanzahlen
    % [C] - Mitte           [E] - Gerade Seitenanzahlen         [CE]    - Seitenmitte, nur gerade Seitenanzahlen
    % [R] - Rechteseite                                         [RO]    - Rechteseite, nur ungerade Seitenanzahlen
    % \fancyhead    Kopfzeile
    % \fancyfoot    Fußzeile
    \fancyhead[L]{\rightmark}
    \fancyhead[R]{Seite \thepage}
    
    % Pixelstärke der Kopfzeilenlinie
    \renewcommand{\headrulewidth}{1pt}
    
    % Setze die Seitenbeginn zurück
    \setcounter{page}{1}
    
    % Importiere weitere .tex Dokumente
    %\include{microservices}
    
    % Glossar
    \printglossaries
    \newpage
    
    % Abbildungsverzeichnis
    \listoffigures
    \newpage
    
    % Literaturverzeichnis
    \printbibliography
   
\end{document}
